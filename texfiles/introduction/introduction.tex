\chapter{Introduction}
\label{chap:introduction}
Here I will write my introduction.
Something about how low clouds in the Arctic have a warming affect as opposed to the cooling effect they have on lower latitudes. This warming is due to the lack of SW radiation flux to reflect, and the emission of LW radiation flux to the ground therefore makes a greater difference.

(Something citing the IPCC report 2013 on the ice conditions in the Arctic, and something about Arctic clouds and radiation?)

Clouds are an important regulator of the Earth's radiation budget. They cover approximately 60\% of the Earth's surface~\citep{Lohmann2005}. It is well know that low clouds have a cooling effect at the surface due to their reflecting of incoming solar radiation. However, the effect is opposite in the Arctic, which I will elaborate on in the theory section. Therefore it is important to investigate the changes in low clouds and their properties in the Arctic.

\section{Main goal}
\section{My contribution}
The findings in my thesis have been achived with some of the most recently developed code (by Greg Thompson) for cloud microphysics and their effects on radiation, in modelling. The results build further on the work of many researchers @name-some-and-cite and contribute and may raise some questions for further research within the field.

\section{Structore of the thesis}
In the following chapter~\ref{chap:background} Background I will present the background for my thesis; what work I hope to compare with and relate my thesis to. Also I will touch upon why the subject of my thesis is important. In chapter~\ref{chap:theory} Theory the most important theory and basic knowledge needed to understand some of the processes in clouds and their possible effect on the sea ice. Chapter~\ref{chap:modmet} Model and methods is where I explain how I have worked with different tools to get the results presented in chapter~\ref{chap:results} which are further discussed in chapter~\ref{chap:discussion}. A summary of main findings and conclusions are presented in the last chapter~\ref{chap:summaryconclusions} Summary and conclusions, before the lists of figures and references at the very end.

\section{Area description}
Here I should describe the area I have studied, and why it was chosen. (figure)
The area was chosen because it had some ice in september, and because there have been research campaigns over the area with focus on clouds in the arctic (which is what I am studying). (show a map..?) Know the correct names of the seas and the land!!