\chapter{Introduction}
\label{chap:introduction}
Here I will write my introduction.
Something about how low clouds in the Arctic have a warming affect as opposed to the cooling effect they have on lower latitudes. This warming is due to the lack of SW radiation flux to reflect, and the emission of LW radiation flux to the ground therefore makes a greater difference.

\section{Main goal}
\section{My contribution}
The findings in my thesis have been achived with some of the most recently developed code (by Greg Thompson) for micro physics in clouds and their effects on radiation, in modelling. The results build further on the work of many researchers @name-some-and-cite and contribute and may raise some questions for further research within the field.

\section{Structore of the thesis}
In the following chapter~\ref{chap:background} Background I willpresent the background for my thesis; what work I hope to compare with and relate my thesis to. Also I will touch upon why the subject of my thesis is important. In chapter~\ref{chap:theory} Theory the most important theory and basic knowledge needed to understand some of the processes in clouds and their possible effect on the sea ice.  Chapter ~\ref{chap:modmet} Model and methods is where I explain how I have worked with different tools to get the results presented in chapter~\ref{chap:results} which are further discussed in chapter~\ref{chap:discussion}. A summary of main findings and conclusions are presented in the last chapter~\ref{chap:summaryconclusions} Summary and conclusions, before the lists of figures and references at the very end.