\chapter{Results and discussion}
\label{chap:results}
In this chapter, the findings made in the thesis are presented. This chapter mostly consists of a discussion of why there is a difference in certain cloud and radiation variables between the run with removed sea ice and the control run, and the run with increased aerosol number concentration and the control run. At the end, there is also a small section on the difference from the control run when both the sea ice is removed and the aerosol number concentration is increased.

The results are discussed separately for removed sea ice and increased aerosol number concentration and for days 1 and 5. Where day 1 represents the closest to an "off line" run, whereas by day 5 the atmosphere has had some time to adjust to the changes.%define offline and online runs when this is mentioned for days 1 and 5

I also try to answer if these results show a warming or cooling effect and if there is reason to believe that changes in sea ice extent and aerosol concentration will further influence the sea ice extent.

The discussion is based on differences in daily averages for the fields. The daily averaged differences have been calculated by subtracting the field from the control run for one time from the same field at the same time from a different run, these differences have then been added together and divided by the number of differences that were added together.(@clarify!)

But first, to have a reference for the differences, the weather and cloud situation for the control run is presented.

%--------------------
\section{Reference figures from the control run}
%--------------------
Figure~\ref{fig:weather} shows the weather situation in the control run for days 1 and 5. The temperature at 2~m height is represented by red contour lines, and the wind speed and direction at 10~m height is shown by the 
wind barbs and their color. Day 1, figure~\ref{subfig:weather_cont_day1} shows weak northerly winds ($\sim$ 5~m/s) bringing cold air (@ input temperature) from the north over the sea ice, and the westerly winds over the ocean south of the sea ice bring moisture to the air over the sea ice which is seen as low stratus in figure~\ref{subfig:cross_LWC_day1}, which shows the LWC in the vertical cross section over the red line in figure~\ref{subfig:cross_line}. The thicker clouds over the island in figure~\ref{subfig:cross_LWC_day1} have been formed due to orographic lifting. The westerly winds over the sea ice towards the mountains at $\sim$76$\degree$N and 112$\degree$W has brought the air to saturation as it is lifted and formed thick clouds over the mountains with LWC of $\sim$0.1$\text{g/m}^3$. From figure~\ref{subfig:cross_IWC_day1}, showing the ice water content (IWC) in the section one can see that the thicker clouds over the mountain also contain ice in the upper part of the clouds, with about 5$\cdot\text{10}^{-3}$~$\text{mg/m}^3$. 

\begin{figure}
    \centering
    \begin{subfigure}{0.48\textwidth}
        \centering
        \includegraphics[width=\textwidth]{results/control/T2UV10_Control_Day1.pdf}
        \caption{Day 1}
        \label{subfig:weather_cont_day1}
    \end{subfigure}
    \begin{subfigure}{0.48\textwidth}
        \centering
        \includegraphics[width=\textwidth]{results/control/T2UV10_Control_Day5.pdf}
        \caption{Day 5}
        \label{subfig:weather_cont_day5}
    \end{subfigure}
    \caption{The temperature and wind pattern for days 1 and 5, from the control run. The temperature at 2~m height is represented by red contour lines and the wind speed and direction at 10~m height is shown by wind barbs and their color, where red is higher wind speed and blue is lower. The shortest tails on the wind barbs indicate 2.5~m/s each, the longer tails indicate 5~m/s and the longest indicate 10~m/s.}
    \label{fig:weather}
\end{figure}

By day 5 the wind direction has changed to south-easterly, see figure~\ref{subfig:weather_cont_day5}, and the clouds in the cross section, figure~\ref{subfig:cross_LWC_Day5} are low stratus over the sea ice, and there is also some thin cloud formation at the mountain, probably formed by weaker orographic lifting. There is no IWC in the section for day 5 (not shown).

\begin{figure}[ht]
    \centering
    \begin{subfigure}{0.48\textwidth}
        \centering
        \includegraphics[width=\textwidth]{results/control/crossSec_LWC_Control_Day1.pdf}
        \caption{LWC, control run, day 1.}
        \label{subfig:cross_LWC_day1}
    \end{subfigure}
    \begin{subfigure}{0.48\textwidth}
        \centering
        \includegraphics[width=\textwidth]{results/control/crossSec_IWC_Control_Day1.pdf}
        \caption{IWC, control run, day1.}
        \label{subfig:cross_IWC_day1}
    \end{subfigure}
    
    \begin{subfigure}{0.48\textwidth}
        \centering
        \includegraphics[width=\textwidth]{results/control/crossSec_LWC_Control_Day5.pdf}
        \caption{LWC, control run, day 5.}
        \label{subfig:cross_LWC_Day5}
    \end{subfigure}
    \begin{subfigure}{0.48\textwidth}
        \centering
        \includegraphics[width=\textwidth]{results/control/crossSec_line.pdf}
        \caption{Line over which the vertical cross sections are made.}
        \label{subfig:cross_line}
    \end{subfigure}
    \caption{Vertical cross sections of averaged liquid ($\text{g/m}^3$) and ice  ($\text{mg/m}^3$) water content, and temperature ($\degree\text{C}$), from the control run, for days 1 and 5. LWC and IWC for day 1 are shown in figures~\ref{subfig:cross_LWC_day1} and~\ref{subfig:cross_IWC_day1} respectively. Figure~\ref{subfig:cross_LWC_Day5} shows the LWC for day 5. The IWC on day 5 was 0 in the section and is not shown. Figure~\ref{subfig:cross_line} shows a map of the area with the ice edge as a black contour line. The terrain height is represented by filled contours and the red line over the sea ice is the line over which the cross sections are made.}
    \label{fig:weather}
\end{figure}

The LWP for days 1 and 5 are shown in figure~\ref{subfig:LWPr1Day1} and~\ref{subfig:LWPr1Day5} with the CDNC (~\ref{subfig:cdnc_cont_Day1} and~\ref{subfig:cdnc_cont_Day5}) and $r_e$ (~\ref{subfig:recloud_r1Day1} and~\ref{subfig:recloud_r1Day5}). One can clearly see that where the LWP is high, so is the CDNC, which is expected based on equation~\ref{eqn:LWC}. The pattern in $r_e$ is @different due to...@

The fluxes of both SW and LW radiation at both the surface and at the top of the atmosphere (TOA) may be partly explained by the clouds, through looking at the LWP. Figure (@radiation) shows the downward SW at the surface (@subfig) and upward at TOA (@subfig) for day 1 and @more radiation and why it looks that way.

The heat fluxes upward at the surface, latent heat (LH) and sensible heat (SH), are also of interest when studying clouds in the Arctic. The fluxes are shown in figure (@figure) for day 1 (LH (@subfig), SH (@subfig)) and day 5 (LH (@subfig), SH (@subfig)). Notice that the LH and SH is lower over the sea ice for both days 1 and 5. This because the sea ice is colder than the ocean, and works as a lid over that part of the ocean, not letting all the heat out.

%--------- Cloud droplet and ice number concentrations
\begin{figure}
	\begin{subfigure}{0.48\textwidth}
		\centering
		\includegraphics[width=\textwidth]{results/control/cdnc_cont_day1.png}
		\caption{Cloud droplet number concentration, averaged over the lower 11 layers and the 1st day.}
		\label{subfig:cdnc_cont_Day1}
	\end{subfigure}
	\begin{subfigure}{0.48\textwidth}
		\centering
		\includegraphics[width=\textwidth]{results/control/cdnc_cont_day5.png}
		\caption{Cloud droplet number concentration, averaged over the lower 11 layers and the 5th day.}
		\label{subfig:cdnc_cont_Day5}
	\end{subfigure}
	\caption{CDNC}
	\label{fig:cdnc}
\end{figure}

\begin{figure}
	\begin{subfigure}{0.48\textwidth}
		\centering
		\includegraphics[width=\textwidth]{results/control/cinc_cont_day1.png}
		\caption{Cloud ice number concentration, plotted over the area, averaged over the lower 11 layers on the 1st day.}
		\label{subfig:cinc_cont_Day1}
	\end{subfigure}
	\begin{subfigure}{0.48\textwidth}
		\centering
		\includegraphics[width=\textwidth]{results/control/cinc_cont_day5.png}
		\caption{Cloud ice number concentration, plotted over the area, averaged over the lower 11 layers on the 5th day.}
		\label{subfig:cinc_cont_Day5}
	\end{subfigure}
	\caption{CINC}
	\label{fig:cloudice}
\end{figure}

%---------- LWP control run, days 1 and 5
\begin{figure}
    \centering
    \begin{subfigure}{0.48\textwidth}
        \centering
        \includegraphics[width=\textwidth]{results/control/LWP_Day1.pdf}
        \caption{Day 1}
        \label{subfig:LWPr1Day1}
    \end{subfigure}
    \begin{subfigure}{0.48\textwidth}
        \centering
        \includegraphics[width=\textwidth]{results/control/LWP_Day5.pdf}
        \caption{Day 5}
        \label{subfig:LWPr1Day5}
    \end{subfigure}
    \caption{The average liquid water path (LWP) for days 1 and 5, including the average value for the field.}
    \label{fig:LWP}
\end{figure}

%----------- Effective radius, days 1 and 5
\begin{figure}[h]
\centering
	\begin{subfigure}{0.48\textwidth}
		\centering
		\includegraphics[width=\textwidth]{results/control/RE_CLOUD_Day1.pdf}
		\caption{The effective radius of cloud droplets average field for day 1 over the lowermost 11 layers.}
		\label{subfig:recloud_r1Day1}
	\end{subfigure}
	\begin{subfigure}{0.48\textwidth}
		\centering
		\includegraphics[width=\textwidth]{results/control/RE_CLOUD_Day5.pdf}
		\caption{The effective radius of cloud droplets average field for day 5 over the lowermost 11 layers.}
		\label{subfig:recloud_r1Day5}
	\end{subfigure}
	\caption{Effective radius of cloud droplets ($\mu\text{m}$), averaged over the lowermost 11 layers for day 1 and 5.}
	\label{fig:recloud_r1}
\end{figure}


---------

%First we want to have something to compare with and to look back to when studying the differences from the control run to the runs with changed sea ice and/or aerosol concentration. I shall include figures for diurnally averaged fields of LWP, $r_e$ of cloud droplet, cloud droplet number concentration (CDNC), both LW and SW up at the TOA and down at the surface. Also the IWP, cloud ice number concentration (CINC), and $r_e$ of snow and ice may be of interest. In addition, the state of the weather situation should be considered when interpreting the results. Therefore maps with wind barbs for the wind at 10~m height and temperature at 2~m are included in this section.

%-------------------------------
\section{Removed sea ice}
%-------------------------------
\subsection{Day 1}
\label{sec:noiceDay1}
The average difference in LWP for NoIce - Control for day 1, shown in figure~\ref{subfig:LWPr2Day1}, is slight for the shole field ($\sim$0.23$\text{g/m}^-2$ increase). But the area of interest in this case is where the sea ice is not present, which it was in the control run. (The area of the sea ice was shown in Chapter~\ref{chap:modmet} in figure~\ref{fig:seaice} and is also included as a black contour in figure~\ref{subfig:cross_line}.) The area where there was sea ice in the control run shows a positive difference in LWP, especially furthest north (bottom right corner of the map) the LWP is significantly higher, >15$\text{g/m}^2$.
\begin{figure}[hb]
\centering
	\begin{subfigure}{0.30\textwidth}
		\centering
		\includegraphics[width=\textwidth]{results/noice/Diff_LWP_Day1NoIce.pdf}
		\caption{LWP, NoIce, day 1}
		\label{subfig:LWPr2Day1}
	\end{subfigure}
	\begin{subfigure}{0.30\textwidth}
		\centering
		\includegraphics[width=\textwidth]{results/noice/diff_NoIce_QNCLOUD_Day1.pdf}
		\caption{CDNC, NoIce, day 1}
		\label{subfig:CDNCr2Day1}
	\end{subfigure}
	\begin{subfigure}{0.30\textwidth}
		\centering
		\includegraphics[width=\textwidth]{results/noice/diff_NoIce_RE_CLOUD_Day1.pdf}
		\caption{$r_e$, NoIce, day 1}
		\label{subfig:recloud_r2Day1}
	\end{subfigure}
\caption{The averaged difference in LWP, CDNC and $r_e$ of cloud droplets (from left to right) for the run with no ice, over the lowermost 11 layers for day 1.}
\label{fig:lwpcdncre_r2Day1}
\end{figure}
This implies that there is a new cloud forming in that area, that could not form when there was sea ice. The removal of the sea ice has allowed for increased evaporation and an increase in latent heat (LH) flux which can be seen from figure~\ref{subfig:lhdiff_r2Day1}, where the area that sea ice was removed from is obvious.
\begin{figure}
\centering
\includegraphics[width=0.5\textwidth]{results/noice/diff_NoIce_LH_Day1.pdf}
\caption{The average difference in LH flux up from the surface at day 2 .}
\label{fig:lh_r2Day1}
\end{figure}
The northernmost part of the study area also has an increase in the cloud droplet number concentration (CDNC), figure~\ref{subfig:cdnc_r2Day1}, in the same area as is covered by the red patch indicating an increase in the LWP, which fits well with equation~\ref{eqn:lwc_prop_cdnc}, which was presented in chapter~\ref{chap:theory}. There the amount of liquid water is proportional to the droplet number concentration, denoted by $N$ in the equation, and the LWP is the vertically integrated LWC. The average increase in the CDNC would be approximately 1 or 2 droplets per cubic centimeter in the northernmost area. The figure shows the numbers with units $10^6/\text{kg}$, which can be approximated to the more common units for CDNC, per cubic centimeter ($\text{cm}^{-3}$). Assuming that the cloud is close enough to the surface to assume a pressure $p=1000\text{hPa}$, and thereby the density to be $\rho_a = 1\text{kg/m}^3=1\text{kg/}10^6\text{cm}^3$, then $\text{CDNC} : 10^6/\text{kg} = \text{cm}^{-3}$. Since this is the average over 11 layers, to a height of about 1600~m, the cloud could be in just a few of those layers, and have a CDNC of approximately 5~$\text{cm}^{-3}$ if it stretches over 3 layers for example. This is then a thin cloud, but it is also averaged over 24 hours, so the cloud could be forming in the later hours of the run. The increase in effective radius in the same area as the LWP is increased also indicates the formation of a new cloud,figure~\ref{fig:recloud_r2Day1}, that could not form in the control run (see figure~\ref{fig:recloud_r1Day1}), whereas now that it has formed, the droplets actually have a radius.
%The small "blob" at 140$^o$W and about 81$^o$N has decreased $r_e$, most likely because the cloud already was saturated in that area, which can be seen from the LWP from the control run, in figure~\ref{fig:LWPr1Day1}. The possible increase in aerosols from the ocean that would then lead to an increase in CCN would make the water in that cloud spread over more CCN, and by that leave the droplets with a smaller $r_e$.


The the two red patches at 80$\degree$N and 140-155$\degree$W in the figure for difference in $r_e$, figure~\ref{subfig:recloud_r2Day1}, is also clear in the difference in LW downward radiation, figure~\ref{subfig:glw_r2Day1} at the ground surface, and can also be slightly recognized as a decrease in SW downward radiation, figure~\ref{subfig:swdown_r2Day1}. The SW radiation flux at ground surface has been reduced due to the increase in LWP, and the more pronounced decrease in SW is clearly recognized with the same shape and size as the northern patch of increase in LWP. This can be explained by equations~\ref{eqn:cloudalbedo} and~\ref{eqn:cloudtau}, where it is clear from equation~\ref{eqn:cloudtau} that the cloud optical depth, $\tau$, increases with LWP, and following equation~\ref{eqn:cloudalbedo} an increase in $\tau$ would also increase the cloud albedo.

\begin{figure}
\centering
	\begin{subfigure}{0.48\textwidth}
		\includegraphics[width=\textwidth]{results/noice/diff_NoIce_SWDOWN_Day1.pdf}
		\caption{The average difference in SW flux down at the surface, day 1.}
		\label{subfig:swdown_r2Day1}
	\end{subfigure}
	\quad
	\begin{subfigure}{0.48\textwidth}
		\centering
		\includegraphics[width=\textwidth]{results/noice/diff_NoIce_SWUPT_Day1.pdf}
		\caption{The average difference in SW flux up at TOA, day 1.}
		\label{subfig:swup_r2Day1}
	\end{subfigure}
	
	\begin{subfigure}{0.48\textwidth}
		\centering
		\includegraphics[width=\textwidth]{results/noice/diff_NoIce_GLW_Day1.pdf}
		\caption{The average difference in LW flux down at the surface, day 1.}
		\label{subfig:glw_r2Day1}
	\end{subfigure}
	\quad
	\begin{subfigure}{0.48\textwidth}
		\centering
		\includegraphics[width=\textwidth]{results/noice/diff_NoIce_LWUPT_Day1.pdf}
		\caption{The average difference in LW flux up at TOA, day 1.}
		\label{subfig:lwup_r2Day1}
	\end{subfigure}
	\caption{The average difference in SW and flux down at the surface and up at TOA, for day 1.}
	\label{fig:radiation_r2Day1}
\end{figure}

The downward LW radiation flux at the surface has been increased due to the increase in LWP, which means that there is more water in the clouds and they emit more LW to the ground. It was shown in Chapter~\ref{chap:theory} that an increase in LWP increases the emissivity of the cloud, shown in equation~\ref{eqn:epsilon_lw}, until the cloud is saturated with respect to LW radiation at about 40-45$\text{g/m}^2$, following figure~\ref{fig:epsalb}.
The slight increase in the LW at TOA -> is it because of increased temperature at the surface when the sea ice is removed?? (@ check it!)

%The LW at the top of the atmosphere (TOA) does not experience such an increase, in fact it experiences a slight decrease. That it doesn't experience the same increase is explained by the Stefan-Boltzmann's law presented in chapter~\ref{chap:theory}, where the flux density emitted by a body, in this case a cloud, is dependent on the temperature and emissivity of the body (equation~\ref{eqn:stefanboltzmann}).
%The temperature contours in figure~\ref{subfig:cross_LWC_day1}, from the control run, show that the temperature decreases with height and that the clouds hold a lower temperature than there is close to the sea ice. In the run with no ice, the situation is the same (see figure~\ref{subfig:cross_LWC_r2day1}), therefore if clouds with lower temperatures than the surface are the source of the LW reaching TOA the LW reaching TOA would be lower.
%The removal of sea ice has a larger effect on lower clouds than on higher clouds, since the increase in evaporation from the surface doesn't reach high up in the troposphere, especially not in the Arctic, due to the static stability of the lower atmosphere in the Arctic(@cite someone?). Also the LWP showed in this study is only for the lowermost 11 layers and can only explain what happens in those layers, it can not be used as a final explanation for radiation changes that are only at the bottom and top of the modeled atmosphere.
%--------------

Of course, the removal of sea ice would reduce the SW at TOA, see figure~\ref{subfig:swup_r2Day1}. The albedo of sea ice varies between 0.5 and 0.9 depending on snow cover and the age of the ice and is typically 0.5-0.7 for bare ice, whereas a typical ocean albedo is 0.06 (cite @ NSIDC). Thus the change in SW at TOA is negative over the area of ocean where there was sea ice in the control run. The increased SW at TOA at 80$\degree$N and 155$\degree$W is because of the cloud forming in that area, see figure~\ref{subfig:swup_r2Day1}, and can be recognized in the increase in $r_e$ in the same place (figure~\ref{subfig:recloud_r2Day1}) which also represent an increase in LWP and reduction in SW at surface and increase of LW at surface. This is due to the enhanced albedo caused by new clouds at that location, since these figures don't show in-cloud changes, simply the difference between two fields.

\begin{figure}
\centering
\includegraphics[width=0.6\textwidth]{results/noice/diff_NoIce_HFX_Day1.pdf}
\caption{The average difference in SH flux up from the surface at day 1.}
\label{fig:sh_r2Day1}
\end{figure}

The heat fluxes are almost unchanged for most of the study area by the removal of sea ice, except for the area where the sea ice has been removed (see figures~\ref{fig:lh_r2Day1} and~\ref{fig:sh_r2Day1}). Especially for the northernmost part of the study area and "sea ice removed area" the fluxes are a lot higher than in the control run. This is not surprising, since one would expect the ocean surface to hold a higher temperature than the sea ice. Also a lot more heat would be released due to evaporation than in the case when sea ice is present. Looking back to figure~\ref{fig:seaice} in Chapter~\ref{chap:modmet}, one can also see that the sea ice fraction is higher further north, which explains the lower LH for that area in the control run.\textbf{@check this!!}

\subsection{Day 5}
The average differences for LWP, CDNC and $r_e$ at day 5 are all negative, see figure~\ref{fig:lwpcdncre_r2Day5}, over the area that had ice in the control run. Thus the clouds making up the LWP in the control run, see figure~\ref{subfig:LWPr1Day5}, have either ceased to exist, been significantly thinned, moved away or turned into ice. The LWP has a negative difference of >30~$\text{g/m}^2$, which means that the LWP, when comparing with the values for that area in the control run (figure~\ref{subfig:LWPr1Day5}) which were around 40-100~$\text{g/m}^2$, there is still around 20-70$\text{g/m}^2$ left. So the clouds have not all ceased to exist. This is supported by the fact that the CDNC in the control run was $\sim$10 to 25~$\text{cm}^{-3}$ and has according to figure~\ref{subfig:CDNCr2Day5} got 3 to >5 droplets~$\text{cm}^{-3}$ less in the run with no ice. Then the clouds in the run with no ice are left with <5 to around 20 droplets~$\text{cm}^{-3}$ which is definitely enough to assume that there are still clouds in the area.
\begin{figure}
\centering
	\begin{subfigure}{0.32\textwidth}
		\centering
		\includegraphics[width=\textwidth]{results/noice/Diff_LWP_Day5NoIce.pdf}
		\caption{LWP, NoIce, day 5}
		\label{subfig:LWPr2Day5}
	\end{subfigure}
	\begin{subfigure}{0.32\textwidth}
		\centering
		\includegraphics[width=\textwidth]{results/noice/diff_NoIce_QNCLOUD_Day5.pdf}
		\caption{CDNC, NoIce, day 5}
		\label{subfig:CDNCr2Day5}
	\end{subfigure}
	\begin{subfigure}{0.32\textwidth}
		\centering
		\includegraphics[width=\textwidth]{results/noice/diff_NoIce_RE_CLOUD_Day5.pdf}
		\caption{$r_e$, NoIce, day 5}
		\label{subfig:recloud_r2Day5}
	\end{subfigure}
\caption{The average difference in LWP, CDNC and $r_e$ of cloud droplets (from left to right) for the run with no ice, over the lowermost 11 layers for day 1.}
\label{fig:lwpcdncre_r2Day5}
\end{figure}
There is hardly any ice at all in the study area in the lowermost 11 layers, much like in the control run (see figure~\ref{subfig:cinc_cont_Day5}), and the IWP is zero (not shown) over the area where there was sea ice, and the area around. The wind pattern (not shown) is very much the same as in the control run (figure~\ref{subfig:weather_cont_day5}), and the chance that the clouds have been moved to a different area is ruled out. Therefore precipitation must have depleted the clouds of some of droplets. The difference in rain (not shown) for the run with no ice compared to the control run is negligible and so snow was found guilty of depleting the clouds. Figure~\ref{fig:snowstory} shows how the cloud that was claimed started to form in day 1 of the run with no ice, in section~\ref{sec:noiceDay1}, as more water vapor and aerosols were made available, develops into a snowing cloud and performs natural cloud seeding by snowing out the other clouds as it travels south-east over the sea ice free area. Figure~\ref{subfig:snowstory_Day1} shows the difference in mixing ratio of snow to air averaged for day 1 over the 11 lowermost layers. The slight increase in mixing ratio of snow is in the same area as the red patch in figure~\ref{subfig:CDNCr2Day5} that was claimed to be a forming cloud. Figure~\ref{subfig:snowstory_Day2} shows that in day 2 the cloud has indeed formed and it starts its journey south-eastward and continues through to day 5, see figure~\ref{subfig:snowstory_Day5} where the positive difference in snow is less pronounced, but still present.
\begin{figure}
\centering
	\begin{subfigure}{0.32\textwidth}
		\centering
		\includegraphics[width=\textwidth]{results/noice/diff_NoIce_qsnow_Day1.pdf}
		\caption{Day 1}
		\label{subfig:snowstory_Day1}
	\end{subfigure}
	\begin{subfigure}{0.32\textwidth}
		\centering
		\includegraphics[width=\textwidth]{results/noice/diff_NoIce_qsnow_Day2.pdf}
		\caption{Day 2}
		\label{subfig:snowstory_Day2}
	\end{subfigure}
	\begin{subfigure}{0.32\textwidth}
		\centering
		\includegraphics[width=\textwidth]{results/noice/diff_NoIce_qsnow_Day3.pdf}
		\caption{Day 3}
	\end{subfigure}

	\begin{subfigure}{0.32\textwidth}
		\centering
		\includegraphics[width=\textwidth]{results/noice/diff_NoIce_qsnow_Day4.pdf}
		\caption{Day 4}
	\end{subfigure}
	\begin{subfigure}{0.32\textwidth}
		\centering
		\includegraphics[width=\textwidth]{results/noice/diff_NoIce_qsnow_Day5.pdf}
		\caption{Day 5}
		\label{subfig:snowstory_Day5}
	\end{subfigure}
\caption{The average difference in mixing ratio of snow to air, over the lowermost 11 layers for days 1 to 5. The difference is calculated by subtracting the field from the control run from the field from the run with no ice.}
\label{fig:snowstory}
\end{figure}
@something about how this change also affect the difference in radiation at both the surface and TOA, in SW and LW.....!

\begin{figure}
\centering
	\begin{subfigure}{0.48\textwidth}
		\includegraphics[width=\textwidth]{results/noice/diff_NoIce_SWDOWN_Day5.pdf}
		\caption{The average difference in SW flux down at the surface, day 5.}
		\label{subfig:swdown_r2Day5}
	\end{subfigure}
	\quad
	\begin{subfigure}{0.48\textwidth}
		\centering
		\includegraphics[width=\textwidth]{results/noice/diff_NoIce_SWUPT_Day5.pdf}
		\caption{The average difference in SW flux up at TOA, day 5.}
		\label{subfig:swup_r2Day5}
	\end{subfigure}
	
	\begin{subfigure}{0.48\textwidth}
		\centering
		\includegraphics[width=\textwidth]{results/noice/diff_NoIce_GLW_Day5.pdf}
		\caption{The average difference in LW flux down at the surface, day 5.}
		\label{subfig:glw_r2Day5}
	\end{subfigure}
	\quad
	\begin{subfigure}{0.48\textwidth}
		\centering
		\includegraphics[width=\textwidth]{results/noice/diff_NoIce_LWUPT_Day5.pdf}
		\caption{The average difference in LW flux up at TOA, day 5.}
		\label{subfig:lwup_r2Day5}
	\end{subfigure}
	\caption{The average difference in SW and flux down at the surface and up at TOA, for day 5.}
	\label{fig:radiation_r2Day5}
\end{figure}
%-------------------------------------
\section{Increased aerosol concentration}
%-------------------------------------
\subsection{Day 1}
The increase in available CCN leads to obvious increases in CDNC and LWP, and the expected reduction in $r_e$, see figure~\ref{fig:cdnclwpre_Aero10}.% If we look back to equation~\ref{eqn:cloudtau}, we see that (@the one that combines LWP, effective radius and cloud optical depth).
\begin{figure}[h!]
\centering
\includegraphics[width=\textwidth]{results/diff_cdnclwpre_Aero10_Day1.png}
\caption{The average differences in $r_e$, LWP and CDNC from left to right, for day 1. (preliminary figure)}
\label{fig:cdnclwpre_Aero10}
\end{figure}

As for the NoIce run, the increase in LWP, in this case a lot higher, leads to an increase in clouds and their reflectance (albedo), therefore the SW at TOA is higher, here the signal is not disrupted by any changes made to the sea ice, so the increase is obvious, and is shown in figure~\ref{fig:swup_down_r3Day2}. Thus the SW at the surface is significantly lower than in the control run. This represents a cooling of (@calculate the flux changes into temperature changes?). The average LW radiation flux at the surface is higher due to the increase in LWP and thereby increased emittance by the clouds.% thicker and denser(is this true) clouds.
\begin{figure}
\centering
\includegraphics[width=\textwidth]{results/aero10/diff_swupdown_Aero10_Day1.png}
\caption{The average difference in SW down at the surface, and up at TOA, from left to right, for day 2. (preliminary figure)}
\label{fig:swup_down_r3Day1}
\end{figure}
The effect on the heat fluxes by increasing the aerosol number concentration is not clear, and probably insignificant (not shown).

\subsection{Day 5}
The LW cloud emissivity is sensitive to an increase in water amount as long as the LWP is less than $\approx$ 40-45 g/m$^2$. It is clear in day 5 from the control run that the LWP was around 60-100 g/m$^2$ in the middle lower area of figure~\ref{fig:LWPr1Day5}.%@ (around these lat and lon?@).
This is also seen in that there is no significant change in LW downward at the surface or upward at the TOA, see figure~\ref{fig:lwup_down_r3Day5}.

\begin{figure}[h!]
\centering
\includegraphics[width=\textwidth]{results/diff_lwupdown_Aero10_Day5.png}
\caption{The average difference in LW downward at the surface and upward at TOA on day 5, from left to right. (preliminary figure)}
\label{fig:lwup_down_r3Day5}
\end{figure}

The area with lack of change in LW up or down is approximately the same area as where there is a negative change in LH and SH upward from the surface over the sea ice, see figure~\ref{fig:lhhfx_r3Day5}. Since there has been no change in LW there is no loss of warming from a decrease in LW reaching the surface, but the change can possibly be explained by looking at the SW radiation. The downward SW at the surface has been significantly decreased as a consequence of the increase in aerosol number concentration, see figure~\ref{fig:swup_down_r3Day5}. The SW radiation has been reflected by the smaller and more numerous droplets. This is known as the Twomey effect, and was described in Chapter~\ref{chap:theory}. 

\begin{figure}[h!]
\centering
\includegraphics[width=0.9\textwidth]{results/diff_swupdown_Aero10_Day5.png}
\caption{The average difference in SW downward at the surface, and upward on TOA at day 5, from left to right. (preliminary figure)}
\label{fig:swup_down_r3Day5}
\end{figure}

\begin{figure}[h!]
\centering
\includegraphics[width=0.9\textwidth]{results/diff_lhhfx_Aero10_Day5.png}
\caption{The average difference in LH and SH upward at the surface on day 5, from left to right. (preliminary figure)}
\label{fig:lhhfx_r3Day5}
\end{figure}

\begin{figure}[h!]
\centering
\includegraphics[width=0.5\textwidth]{results/diff_skintemp_Aero10_Day5.png}
\caption{The average difference in skin temperature (the temperature of the surface), for day 5.}
\label{fig:skintemp_r3Day5}
\end{figure}
The albedo of sea ice is typically 0.5-0.7 which means that a fraction of the incident SW radiation is absorbed. Since the amount of incident SW radiation at the surface has been reduced by the cloud cover, the absorbed radiation is less than for a higher incident amount. The ice therefore has a lower temperature to give off SH with. 

%Look at temperature changes at the surface. Over ice and ocean.. 
The skin temperature, figure~\ref{fig:skintemp_r3Day5}, for the domain shows a small decrease in the same area as where there is less sensible and latent heat release.

Also the dynamics over ice and the ocean are different, so this could have an effect. The cold air over the ice may be intensified, while the sea surface temperature (SST) is the same for all the runs and there is no coupling with the ocean. There will therefore be no changes in the SSTs that could have an effect on the dynamics over the ocean. So the response may be smaller there...?

%Må ha skala på labelbar som gjør det leselig, og som fremhever de små forskjellene -- for de kan ha stor betydning..!

% Må også oppgi gjennomsnittsverdi for hele feltet (for å hjelpe leseren) og minimums og maksimumsverdiene?, siden de ikke er med på labalbaren...

%Hvordan de forskjellige tingene påvirker hverandre må være dekket i teorien.!

%Jon Egill har avtalt eksamen 19. juni. Erik Berge er sensor! Hvem er intern?
