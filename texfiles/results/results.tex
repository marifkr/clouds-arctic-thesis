\chapter{Results and discussion}
\label{chap:results}
\section{one for each run with belonging discussion}
For now we will only look at the difference in some variables from two runs. The control run and the run with no ice (and unchanged initial aerosol concentrations).
%First lets have a look at the.... changes in something from the control run when the sea ice was removed.
Here I have simply included some figures that show the difference between the two runs for some variables at 20:00UTC and Eta level 5 ($\sim$ 370m, from table~\ref{tab:etaheights}) for cloud water mixing ratio (see figure~\ref{fig:qcloud}) and effective radius (see figure~\ref{fig:re_cloud}), since these two variables also vary with height. But not discussed them, this due to an NCL memory issue and lack of brain capacity at late hours. A better presentation of results (more than just these figures) will definitely be included in the next edition.

\begin{figure}
\centering
\includegraphics[width=0.95\textwidth]{../diff_SWUPT.png}
\caption{Difference in upward SW radiation flux at top of the atmosphere (TOA). How much more (red) or less (blue) SW radiation reaches the TOA in the no ice run, compared to the control run.}
\label{fig:swupt}
\end{figure}

\begin{figure}
\centering
\includegraphics[width=0.95\textwidth]{../diff_LWUPT.png}
\caption{Difference in upward LW radiation flux at TOA. How much more (red) or less (blue) LW radiation reaches the TOA in the run without ice, compared to the control run.}
\label{fig:lwupt}
\end{figure}

\begin{figure}
\centering
\includegraphics[width=0.95\textwidth]{../diff_SWDOWN.png}
\caption{Difference in downward SW radiation flux at ground surface. How much more (red) or less (blue) SW radiation reaches the surface in the run without ice, compared to the control run.}
\label{fig:swdown}
\end{figure}

\begin{figure}
\centering
\includegraphics[width=0.95\textwidth]{../diff_GLW.png}
\caption{Difference in downward LW radiation flux at ground surface. How much more (red) or less (blue) LW radiation reaches the surface in the run without ice, compared to the control run.}
\label{fig:glw}
\end{figure}

\begin{figure}
\centering
\includegraphics[width=0.95\textwidth]{../diff_LH.png}
\caption{Difference in latent heat flux at surface. How much higher (red) or lower (blue) the flux is in the run without ice, compared to the control run.}
\label{fig:lh}
\end{figure}

\begin{figure}
\centering
\includegraphics[width=0.95\textwidth]{../diff_HFX.png}
\caption{Difference in sensible heat flux at surface. How much higher (red) or lower (blue) the flux is in the run without ice, compared to the control run.}
\label{fig:hfx}
\end{figure}

\begin{figure}
\centering
\includegraphics[width=0.95\textwidth]{../diff_QCLOUD.png}
\caption{Difference in cloud water mixing ratio, how much larger (red) or smaller (blue) the droplets were in the run without ice, compared to the control run.}
\label{fig:qcloud}
\end{figure}

\begin{figure}
\centering
\includegraphics[width=0.95\textwidth]{../diff_RE_CLOUD.png}
\caption{Difference in effective radius. How much larger (red) or smaller (blue) the droplets were in the run without ice, compared to the control run.}
\label{fig:re_cloud}
\end{figure}

%Her kan det være interessant å se på\\
%Differanser i\\
%- SW fluks\\
%- LW fluks\\
%- LH \\
%- SH, følbar varme... har jeg det?\\
%- Dråpestørrelse\\ level 5
%- Skyvannmengde\\ level 5

%Må få laget disse figurene sånn at Jon Egill og Kari kan se på dem, og sånn at jeg kan tenke på dem og reflektere og diskutere med meg selv!

%Må ha skala på labelbar som gjør det leselig, og som fremhever de små forskjellene -- for de kan ha stor betydning..!

%Hvordan de forskjellige tingene påvirker hverandre må være dekket i teorien.!

%Jon Egill hadde tenkt eksamen 22. juni.....

%- HFX, sensible heat flux at the surface. Upward heat flux at surface. W/m2