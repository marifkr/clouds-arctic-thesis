\chapter*{Abstract}
The influence of diminishing sea ice and increased aerosol number concentration on low clouds over the Beaufort Sea, north of Canada and Alaska, has been investigated by use of a formulation of the Weather Research and Forecasting (WRF) model called the Advanced Research WRF (ARW).

The model was run for the five first days of September 2012, of which the discussion focuses on days 1 and 5. There the first day is most like an off-line run, representing near instantaneous changes in clouds and radiation due to ice removal and aerosol number concentration increase, whereas by day five the atmosphere has had time to adapt to the changes that were imposed at the start of the first day.

The near instantaneous changes as a consequence of removal of sea ice were negative in upward shortwave radiation (SW) at the top of the atmosphere (TOA) due to decreased surface albedo. There were also signs of new clouds forming, indicated by an increase in liquid water path (LWP) of 15~$\text{g/m}^2$. As the atmosphere has had time to adapt to the changed sea ice extent, increased precipitation release dominates over increased moisture supply from the open ocean, leading to an average decrease in LWP, -2.3~$\text{g/m}^2$. %Due to increased temperature in the clouds, the downward longwave (LW) at the surface is increased by $\sim$5~$\text{W/m}^2$ for parts of the newly opened ocean, despite the decrease in LWP. Still, 
~The average change in downward LW for the domain was -0.15~$\text{W/m}^2$.

The near instantaneous changes following an increase in aerosol number concentration are increases in LWP and cloud droplet number concentration (CDNC), and a decrease in cloud droplet size. These more numerous and smaller droplets increase the albedo of the clouds, known as the first indirect effect. The increase in LWP indicates that the clouds are also denser, which is known as the second indirect effect. Both these effects reduce the downward SW at the surface, giving a change of -9.2~$\text{W/m}^2$. As the atmosphere has had time to adapt, the cooling effect from reduced downward SW is evident in the surface temperature and heat fluxes, as they decrease.

In this study, initially high LWP (40-300~$\text{g/m}^2$) weakens the enhancement of LW down at the surface, since the clouds some times are saturated with respect to LW, and the SW signal dominate following sea ice decrease and aerosol number concentration increase in September. A positive feedback between changes in Arctic stratus and changes in Arctic sea ice extent is not confirmed by this study.