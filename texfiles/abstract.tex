\chapter*{Abstract}
The influence of diminishing sea ice and increased aerosol number concentration on low clouds over the Beaufort Sea, north of Canada and Alaska, has been investigated by use of a formulation of the Weather Research and Forecasting (WRF) model called the Advanced Research WRF (ARW).

The model was run for the five first days of September 2012, of which the discussion focuses on days 1 and 5. There the first day is most like and off line run, representing near instantaneous changes in clouds and radiation due to ice removal and aerosol number concentration increase, whereas by the last day the atmosphere has had time to adapt to the changes that were implemented at the start of the first day.

The near instantaneous changes as a consequence of removal of sea ice were negative changes in upward shortwave radiation (SW) at the top of the atmosphere (TOA) and downward SW at the surface. There were also signs of new clouds forming, indicated by a small increase in liquid water path (LWP) 0.23~$\text{g/m}^2$. As the atmosphere has had time to adapt to the changed sea ice extent, new clouds have formed and precipitated, also causing other clouds to precipitate. Thus leading to an average decrease in LWP, -2.3~$\text{g/m}^2$.

The near instantaneous changes following an increase in aerosol number concentration are increases in LWP and cloud droplet number concentration (CDNC), and a decrease in cloud droplet size. These more numerous and smaller droplets increase the albedo of the clouds, known as the first indirect effect. The increase in LWP indicates that the clouds are also denser and longer liver, which is known as the second indirect effect. Both these effects reduce the downward SW at the surface, giving a change of -9.2~$\text{W/m}^2$. They also increase the SW up at TOA by 7.3~$\text{W/m}^2$. As the atmosphere has had time to adapt, the cooling effect from reduced downward SW is evident in the surface temperature and heat fluxes, as they decrease.

In this study, initially high LWP (40-300~$\text{g/m}^2$) weakens the enhancement of LW down at the surface, since the clouds some times are saturated with respect to LW. Thus a positive feedback between changes in Arctic stratus and changes in Arctic sea ice extent may not be confirmed.