\chapter{Background}%Background and motivation?
\label{chap:background}
In this chapter I shall present some recent literature on the subject, that forms a basis for the study presented in this thesis.

A study by~\citet{Schweiger2008} investigated the connection between sea ice variability and cloud cover over over the Arctic seas during autumn. They analyzed the ERA-40 re-analysis products and the Television and Infrared Observation Satellite (TIROS) Operational Vertical Sounder (TOVS) Polar Pathfonder datasets~\citep{Schweiger2008}. They found that that sea ice retreat was linked to a decrease in low-level (surface to \tilde{ }1.9~km) cloud amount and an increase in midlevel (\tilde{ }1.9 to 6.1~km) clouds. They state that the decrease in static stability and deepening of the atmospheric boundary layer, following ice retreat, contribute to the rise in cloud level. 
 %the 40-yr European Centre for Medium-Range Weather Forecasts (ECMWF) Re-Analysis (ERA-40) products

The study by~\citet{Vavrus2010} investigated the behaviour of clouds, during intervals of rapid sea ice loss in the Arctic in the 21st century. The study was done by use of the Community Climate System Model (CCSM3). Their results support that cloud changes appear to accelerate rapid loss of sea ice in autumn, and possibly in winter. They also conclude that "the trends in total cloudiness during rapid ice loss events are explained almost entirely by low-level clouds" and that "a positive feeback from primarily low cloud changes amid a warming climate".

\citet{Kay2009} combined satellite data and complementary data sets to study the Arctic cloud and atmospheric structure during summer and early Autumn over the years 2006-2008. This covers the at the time record low sea ice extent from 2007. In contrast to the study by~\citet{Schweiger2008} they found more low-level cloud. There  are reasons to belive that the observations used in their study are more accurate than the re-analysis used in~\citet{Schweiger2008}.


\citet{Eastman2010a} analyzed visual cloud reports from the Arctic for year-to-year variations 

The study by~\citet{Palm2010} using satellite and lidar data and found that areas of open water were associated with greater polar cloud fraction. %from the Ice Cloud, and Land Elevation Satellite (ICESat) and the Cloud-Aerosol Lidar and Infrared Pathfinder Satellite Observation (CALIPSO)  found that areas of open water were associated with greater polar cloud fraction.


A common uncertainty and missing link in a few of these studies is that they did not look at liquid water content, effective radii and other paramterers affecting the radiative properties of the clouds. In this study, that is what I want to look in to, how these properties are influenced by the changing sea ice, and if the cloud enhance the sea ice melt.

Some new (and older) research that makes my thesis interesting and relevant to the field. No one has done this study with a weather forecasting model, it is interesting to look at changes in parameters in similar meteorological conditions, the same initial and boundary conditions in every run.
Describe some of the work done by \citet{Palm2010, Wu2012} and others on autumn low clouds in the Arctic to emphasize the importance of my work.

%Is my work new thinking and does it seem to be very extensive?

%Should this whole chapter be a part of introduction..?

%Previous studies, shed light on why my study is interesting and may be of importance.

%Palm, Kay and Gettleman, Eastman and Warren according to the IPCC report!