\chapter{Model and methods}
\label{chap:modmet}
\section{Model description}
\label{sec:modeldes}
To produce results for the thesis, a formulation of the Weather Research and Forecasting (WRF) Model called the Advanced Research WRF (ARW) has been used, version 3.6.1. The model is developed at the National Centre for Atmospheric Research (NCAR) in Boulder, Colorado. The ARW model is the first fully compressible conservative form nonhydrostatic model designed for both research and operational numerical weather prediction (NWP) applications \citep{Skamarock2008}.

\section{Model setup}
\label{sec:modelset}
I run ARW with a horisontal resolution of 4 km, and 72 vertical layers. This resolution is sufficient to resolve clouds @citation.
The vertical layers in the ARW model are called eta levels. These levels have uneven vertical spacing, defined by @insertequation @citaion. Since the level is dependent on pressure, the height varies in both time and space. Consequently the levels in the lower troposphere are closer to each other than higher up in the troposphere. Therefore the low clouds in the area can be resolved. (@How close? what heights??)

@Area description. Sea names: Beaufort and ??. By Canada and Alaska, this is because data from the area has been used for research by others @citations. The area is not ice free any part of the year @cite, and provides a good place to simulate cloud and sea ice interaction.

The sea ice in the area was removed by editing the input file constructed? built? by WPS and real.exe, to get results to compare with results from runs with ice.

From diminishing sea ice we might experience an increase in sea traffic, which would lead to an increase in aerosol content in otherwise clean air @citation. To include increase in aerosol concentrations due to lack of sea ice I used the microphysics scheme developed by Greg Thompson and Trude Eidhammer described in \cite{Thompson2014}.

The scheme uses a monthly mean for aerosol number concentrations derived from multi-year (2001-2007) global model simulations @citationColarco2010??(det har de cita) in which particles and their precursors are emitted by natural and anthropogenic sources and are explicitly modeled with multiple size bins for multiple species of aerosols by the Goddard Chemistry Aerosol Radiation and Transport (GOCART) model (@desiterteGinoux2001).

Choice of schemes and reasons should be presented. As should hos WPS and real.exe and wrf.exe works. At least short about what they do and contribute with to get to the end results.
\section{Model runs}
\subsection{Manipulation of input files}
Elaborate on removal or placing of sea ice. Elaborate on multiplying the aerosol number concentration with a factor 10. By use of ncap2 from NetCDF (NCO).
\section{Input data}
E.... (ECMWF) ERA-Interim reanalysis... used as input for initial and boundary?? conditions. Downloaded by python scripts provided by Anne (more names) at the IT-help ??? at section for Meteorology and Oceanography at the University of Oslo.
\section{Processing of the results}
Figures presented in my thesis, I made (unless other is stated) by use of National Centre for Atmospheric Research (NCAR) Command Language (NCL), with a lot of help and inspiration from the example scripts for WRF-users available at (URL for examples).