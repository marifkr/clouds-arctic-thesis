\chapter{Theory}
\label{chap:theory}
\section{Arctic clouds}
The air in the Arctic is very stable in winter (polar night) and clean as there are not many sources for pollution. In Autumn the sea ice extent reaches a minimum after the summer melting and leave open water to influence low clouds and their properties. 

Low clouds have bases below 2000 m. Stratus (St) are layered clouds that form when extensive areas of stable air are lifted. Stratus clouds are normally between 0.5 and 1~km thick, whereas they can be several km wide. (@citeAguadoBurtpage188?) But how high does the top of a cloud with a low base reach??

\section{Radiation and clouds}
How clouds scatter and absorb SW and LW radiation.
Explain something about blackbodies, clouds and blackbodies? Stefan–Boltzmanns law states that the flux density emitted by a blackbody is proportional to the fourth power of the absolute temperature @citeLiou2002page12. For a greybody, like a cloud, the equation can be written
\begin{equation}
F = \epsilon \sigma T^4
\end{equation}
where the emissivity of the greybody, $\epsilon(units?)$, is included. $F(units)$ is the flux density emitted by the greybody, and $\sigma = 5.67\cdot 10^{-8} Jm^{-2}sec^{-1}deg^{-4}$ is the Stefan–Boltzmann constant.


Write about optical depth from Wallace and Hobbs: Normal optical depth or optical thickness, $\tau_{\lambda}$ is a measure of the cumulative depletion that a beam of radiation directed straight downward (zenith angle $\theta = 0$) would experience in passing through a defined layer \citep{WallaceHobbs2006}.
\begin{equation}
\tau_{\lambda} = \int_z^{\infty} k_{\lambda} \rho r dz
\end{equation}
where $k_{\lambda}$ is the mass absorption coefficient, which has units of $m^2~kg^{-1}$, $\rho$ is the density of air, which has units of $kg~m^{-3}$, and $r$ is the mass of the absorbing gas per unit mass of air.

Meg: The optical depth will change with changes in aerosol number concentrations (aerosol content?) and changes in clouds and their properties. For instance if a cloud has many small droplets, the optical depth will be higher. Where as  fewer cloud droplets will yield a lower optical depth, resulting in more SW radiation reaching the ground — possibly having a warming effect on the area. 



How do clouds reflect radiation? What is the effect of more water? Or more ice? What about the droplet size? (effective radius)\\
How do clouds absorb and emit radiation? Effect of more or less water or ice? Droplet size?

Ice is more effective in reflecting SW than water. Snow has a higher albedo than rain. Is there any use in presenting some albedo values for water, snow, ice? Or open water versus sea ice? New ice versus old sea ice?

\section{Aerosol affect on clouds}
\subsection{The first indirect effect}
Twomey 1974?\\
By increasing droplet concentration and hence the optical thickness of a cloud, pollution acts to increase the reflectance clouds~\citep{Twomey1977}. 
The optical thickness is increased when the number of CCN is increased. Although the changes are small, the long term effects on climate can be profound~\citep{Twomey1974}.


\subsection{The second indirect effect}
Albrecht 1989



