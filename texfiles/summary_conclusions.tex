\chapter{Summary and Conclusions}
\label{chap:summaryconclusions}
In this thesis the cloud radiative response to removal of sea ice and increased aerosol number concentration was studied by use of the ARW model. The model was run over 5 days in early autumn, with and without sea ice, and with and without increased aerosol number concentration. The study area covers the Beaufort Sea. The hypothesis was that there could be a positive feedback between the declining areal Arctic sea ice extent (eg. National Snow and Ice Data Center~\citep{NSIDC}) and radiative response of low Arctic stratus in autumn. Studies by @(cite some) have found that the lack of sea ice in early autumn has led to an increase in low cloud amount. @someone also found that the clouds had longer lifetimes when there was no sea ice beneath, due to the enhanced evaporation from the open ocean. 

@the aforementioned studies did not look at the microphysical changes in the clouds, which has been the focus of this study. The response of clouds to removal of sea ice and increased aerosol number concentration has been studied both separately and combined, for both the the first day of the run, which acts almost as an off-line run, and the last day of the run, when the atmosphere has had time to adapt to the changes implemented on the start of the first day.

Some key findings in the thesis are:
\begin{itemize}
\item Here I will list some results as 
\item bullet points
\item for clarity, and so that it is easy to follow
\end{itemize}

Then I shall interpret the main findings and put them into context. What do the results mean, and do they answer the question in the title? 

Here I also want to mention shortcomings of this study and maybe some shortcomings of the model.

Outlook: For future research I should mention something that could be done.



%Stuff to do:\\
%-- Kort oppsummering av hva som er gjort og hvorfor\\
%-- Kort oppsummering av resultater\\
%-- Kort tolkning av resultatene\\
%-- Sluttord (f.eks. videre forskning eller annet perspektiv)