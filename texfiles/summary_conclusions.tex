\chapter{Summary and concluding remarks}
\label{chap:summaryconclusions}
In this thesis the cloud radiative response to removal of sea ice and increased aerosol number concentration was studied by use of a formulation of the Weather Research and Forecasting (WRF) model called the Advanced Research WRF (ARW). The model was run for the first 5 days of September 2012, which is the most recent year of record low sea ice extent~\citep{NSIDC}. The model was run with and without sea ice, and with and without increased aerosol number concentration. The study area covers the Beaufort Sea, north of Canada and Alaska. The hypothesis was that there could be a positive feedback between the declining areal Arctic sea ice extent (eg. National Snow and Ice Data Center~\citep{NSIDC}) and radiative response of low Arctic stratus in autumn. Studies by~\citet{Eastman2010a,Kay2009,Palm2010} have found that the lack of sea ice in early autumn has led to an increase in low cloud amount. The question is then if these clouds have a warming effect at the surface. This could enhance sea ice melt and/or delay freezing, both of which would further decrease the sea ice extent. 

The aforementioned studies did not look at the microphysical changes in the clouds, which has been the focus of this study. The response, and radiative effects of clouds to removal of sea ice and increased aerosol number concentration has been studied both separately and combined, for both the the first day of the run, which acts almost as an off-line run showing near instantaneous changes, and the last day of the run, when the atmosphere has had time to adapt to the changes imposed on the start of the first day.

Summary of results:
\begin{itemize}
\item Near instantaneous changes as a consequence of removed sea ice are increased surface heat fluxes, increased surface temperature and emission of water-friendly aerosols from the newly opened ocean. The enhanced evaporation from the ocean together with convection leads to formation of new clouds, with an increases in liquid water path (LWP) of 15~$\text{g/m}^2$ and cloud droplet number concentration (CDNC) of 1 or 2~$\text{cm}^{-3}$ in the cloud forming area (increase in LWP and CDNC on average were 0.2~$\text{g/m}^2$ and 0.004~$\text{cm}^{-3}$, respectively). New clouds give increased downwelling LW radiation at the surface $\sim$14~$\text{W/m}^2$ (0.27~$\text{W/m}^2$ on average). The higher temperature of the surface increases the daily time averaged upwelling LW radiation at the top of the atmosphere (TOA) by 0.1~$\text{W/m}^2$. Removing the sea ice significantly reduces the albedo, thus decreasing both upwelling SW at TOA, and downwelling SW at the surface, which is no longer reflected between sea ice at the surface and the clouds above.
\item Near instantaneous changes as a consequence of increased aerosol number concentration are increased LWP (11~$\text{g/m}^2$) and CDNC (16~$\text{cm}^{-3}$) and decreased effective radius, $r_e$ (-0.5~$\mu\text{m}$). The combination of increased CDNC and decreased $r_e$ increases the cloud albedo, and is known as the first indirect effect. The increase in daily time averaged LWP indicates that the clouds are denser, enhancing the reflectance of SW, which is known as the second indirect effect. The SW up at TOA is increased by 7.3~$\text{W/m}^2$, and down at the surface it is reduced by 9.5~$\text{W/m}^2$. The increase in LW down at the surface is 2.3~$\text{W/m}^2$. Thus, the cooling effect by reduced downward SW overpowers the warming effect from enhanced LW down at the surface. This is also seen in the decrease in temperature of the surface (-0.02~$\degree$C) and reduction in sensible heat flux (-0.4~$\text{W/m}^2$).
\item When increase in aerosol number concentration and removal of sea ice is combined, the near instantaneous changes in LWP, CDNC and $r_e$ are the same as when the sea ice is unchanged. The SW down at the surface is reduced by 11.6~$\text{W/m}^2$, and up at TOA it is increased by 5.1~$\text{W/m}^2$. Thus the combined removal of sea ice and increase in aerosols have a greater cooling effect at the surface through reduction in SW. On the other hand, the albedo of the ocean is significantly lower, meaning that more of the downward SW at the surface is absorbed, which has a warming effect.
\item A few days after removal of the sea ice, the atmosphere has had time to adapt and in this study the increased surface temperatures and surface heat fluxes have made new clouds form. These clouds precipitate and cause other clouds to precipitate, hence leading to an average reduction in LWP (-2.3~$\text{g/m}^2$), CDNC (-0.45~$\text{cm}^{-3}$) and $r_e$ (-0.16~$\mu\text{m}$).
\item The effect of increased aerosol number concentration after a few days is similar to the first day. The clouds are denser and longer lived, but the warming effect in LW (1.8~$\text{W/m}^2$) down at the surface is overpowered by the cooling effect in SW (-8.6$\text{W/m}^2$) at the surface.
\item When increase in aerosol number concentration and removal of sea ice is combined, the increase in LWP is still evident, but lower than for just the aerosol increase, 16.7~$\text{W/m}^2$. This is because of the precipitation caused by convention due to increased surface temperatures and increased surface heat fluxes. The changes in radiation down at the surface is 1.8~$\text{W/m}^2$ in the LW, and -8.6~$\text{W/m}^2$ in the SW. Thus indicating a cooling effect at the surface.
\end{itemize}

In the results presented in this thesis the LWP in the control run was already high, with 40-150~$\text{g/m}^2$ for most of the study area and values up to 200 and 300~$\text{g/m}^2$ at the most. This means that the clouds were saturated with respect to LW radiation for large parts of the study area. Thus the warming effect from enhanced downwelling LW radiation is overpowered by the cooling effect by reflection of SW radiation. Therefore a positive feedback between Arctic stratus and changes in Arctic sea ice extent can not be confirmed based on this study.

There are a couple of possible errors in the way the results are presented, which should be mentioned. First, I have studied only the clouds in the lower 1800~m of the atmosphere, but looked at changes in SW and LW radiation that are affected by all the clouds in the troposphere. Secondly, the emissivity and albedo of the clouds do not vary linearly with changes in LWP. This means that there could be variations within a day in LWP that is not detected by the difference in daily averages, but gives a signal in the difference in SW or LW radiation that are detected.

To further study the possibility of such a feedback loop, it would be interesting to look at changes in radiative cloud properties for several years with different sea ice extents, to establish a statistically significant set of results. The effects on radiative cloud properties by decreasing sea ice extent and increased aerosol number concentration should also be studied for other areas in the Arctic. The results should also be compared with observations, if possible.

Using a weather research model, as opposed to a climate model, allows for a detailed study of the low clouds, with high resolution horizontally (grid boxes of size4~km $\times$4~km) and vertically (72 layers up to 10hPa). This resolution is also better than many observations. Although there have been field campaigns focusing on clouds (see Chapter~\ref{chap:introduction}), observations in the Arctic can be challenging and costly, due to a long dark season and a lot of open water, often meaning that air planes or ships must be involved in collecting the data. When that is said, observations do at least show real life changes, whereas a model can only simulate the changes.