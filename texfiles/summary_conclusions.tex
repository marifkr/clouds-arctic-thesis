\chapter{Summary and concluding remarks}
\label{chap:summaryconclusions}
In this thesis the cloud radiative response to removal of sea ice and increased aerosol number concentration was studied by use of a formulation of the Weather Research and Forecasting (WRF) model called the Advanced Research WRF (ARW). The model was run for the first 5 days of September 2012, which is the most recent year of record low sea ice extent~\citep{NSIDC}. The model was run with and without sea ice, and with and without increased aerosol number concentration. The study area covers the Beaufort Sea, north of Canada and Alaska. The hypothesis was that there could be a positive feedback between the declining areal Arctic sea ice extent (eg. National Snow and Ice Data Center~\citep{NSIDC}) and radiative response of low Arctic stratus in autumn. Studies by~\citet{Eastman2010a,Kay2009,Palm2010} have found that the lack of sea ice in early autumn has led to an increase in low cloud amount. The question is then if these clouds have a warming effect at the surface. This could enhance sea ice melt and/or delay freezing, both of which would further decrease the sea ice extent. 

The aforementioned studies did not look at the microphysical changes in the clouds, which has been the focus of this study. The response, and radiative effects of clouds to removal of sea ice and increased aerosol number concentration has been studied both separately and combined, for both the the first day of the run, which acts almost as an off-line run showing near instantaneous changes, and the last day of the run, when the atmosphere has had time to adapt to the changes implemented on the start of the first day.

Summary of results:
\begin{itemize}
\item Near instantaneous changes as a consequence of removed sea ice are increased surface heat fluxes, increased surface temperature and emission of water-friendly aerosols from the newly opened ocean. The enhanced evaporation from the ocean together with convection leads to formation of new clouds, with small increases in liquid water path (LWP) (0.2~$\text{g/m}^2$) and cloud droplet number concentration (CDNC) (0.004~$\text{cm}^{-3}$). New clouds give increased downwelling LW radiation at the surface $\sim$0.27~$\text{W/m}^2$. The higher temperature of the surface increases the daily time averaged upwelling LW radiation at the top of the atmosphere (TOA) by 0.1~$\text{W/m}^2$. Removing the sea ice significantly reduces the albedo, thus decreasing both upwelling SW at TOA, and downwelling SW at the surface, which is no longer reflected between sea ice at the surface and the clouds above.
\item Near instantaneous changes as a consequence of increased aerosol number concentration are increased LWP (11~$\text{g/m}^2$) and CDNC (16~$\text{cm}^{-3}$) and decreased effective radius, $r_e$ (-0.5~$\mu\text{m}$). The combination of increased CDNC and decreased $r_e$ increases the cloud albedo, and is known as the first indirect effect. The increase in daily time averaged LWP indicates that the clouds are denser and longer lived, enhancing the reflectance of SW, which is known as the second indirect effect. The SW up at TOA is increased by 7.3~$\text{W/m}^2$, and down at the surface it is reduced by 9.5~$\text{W/m}^2$. The increase in LW down at the surface is 2.3~$\text{W/m}^2$. Thus, the cooling effect by reduced downward SW wins over the warming effect from enhanced LW down at the surface. This is also seen in the decrease in temperature of the surface (skin temperature), -0.02~$\degree$C, and reduction in sensible heat flux, -0.4~$\text{W/m}^2$.
\item When increase in aerosol number concentration and removal of sea ice is combined, the near instantaneous changes in LWP, CDNC and $r_e$ are the same as when the sea ice is unchanged. The SW down at the surface is reduced by 11.6~$\text{W/m}^2$, and up at TOA it is increased by 5.1~$\text{W/m}^2$. Thus the combined removal of sea ice and increase in aerosols have a greater cooling effect at the surface through reduction in SW. On the other hand, the albedo of the ocean is significantly lower, meaning that more of the downward SW at the surface is absorbed, which has a warming effect.
\item A few days after removal of sea ice the atmosphere has had time to adapt and in this study the increased surface temperatures and surface heat fluxes have made new clouds form, which precipitate and also cause other clouds to precipitate leading to a reduction in LWP (-2.3~$\text{g/m}^2$), CDNC (-0.45~$\text{cm}^{-3}$) and $r_e$ (-0.16~$\mu\text{m}$).
\item 
\item 
\end{itemize}

These results

Using a weather research model, as opposed to a climate model, allows for a detailed study of the low clouds, with high resolution both in the horizontal and in the vertical. This resolution is also better than many observations. Although there have been a few field campaigns (see Chapter~\ref{chap:introduction}), observations in the Arctic can be challenging and costly, due to a long dark season and a lot of open water, often meaning that air planes or ships must be involved in collecting the data.

On the other hand, it is still a model, and nothing

@Outlook: For future research I should mention something that could be done.


%Stuff to do:\\
%-- Kort oppsummering av hva som er gjort og hvorfor\\
%-- Kort oppsummering av resultater\\
%-- Kort tolkning av resultatene\\
%-- Sluttord (f.eks. videre forskning eller annet perspektiv)